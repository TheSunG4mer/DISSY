\section{Week 41 - Key Management and Infrastructure}

\subsection*{Exercises}
9.1; 9.2; 9.4; 9.5; 9.9; 9.6; 9.7; 9.8; 9.10
\medbreak

Many of the discussion questions for this week are highly subjective and context dependent, with the exercises and solution guide simply providing a starting point for discussion.

\Exercise{9.1}
\begin{itemize}
    \item There is no guarantee that the message came from A, anyone can use the public key of B. 
    \item B provides a decryption oracle, if we intercept the return message, we can ask for this to be decrypted and receive R (by starting our own protocol), while B obtains some nonsense key from decrypting R.
    With R an adversary can produce and desired MACs and decrypt all messsages.
\end{itemize}

\Exercise{9.2}
The Needham-Schroeder protocol as described in the book is shown below.
\begin{center}
    \fbox{
        \procedure{}{
            \textbf{Alice} \> \> \textbf{Bob}\\ [0.1\baselineskip][\hline]
            \\ [-0.5\baselineskip]
            \text{Choose }  n_A\\
             \>
            \sendmessageright*{{E_{pk_B}(ID_A,n_A)}} 
            \> \\%ID_A, n_A\\
            \> \>  \text{Choose } n_B\\
            \> \sendmessageleft*{E_{pk_A}(n_A,n_B)} \>\\
            \text{Check }n_A \\
            \> \sendmessageright*{E_{pk_B}(n_B)} \\
            \> \> \text{Check } n_B\\
            \\
            k = \textrm{KDF}(n_A,n_B) \> \> k = \textrm{KDF}(n_A,n_B) \\
        }
    }     
\end{center}
The MITM attack:
\begin{figure}[H]
    \centering
    \fbox{
        \procedure{}{
            \textbf{Alice} \> \> \textbf{Eve}\> \>\textbf{Bob} \\ 
            [0.1\baselineskip][\hline]
            \\ [-0.5\baselineskip]
            \text{Choose }  n_A\\
             \>
            \sendmessageright*{{E_{pk_E}(ID_A,n_A)}} 
            \>\text{switch keys}  \> \sendmessageright*{{E_{pk_B}(ID_A,n_A)}} \\%ID_A, n_A\\
            \> \>  \> \>\text{Choose } n_B\\
            \> \sendmessageleft*{E_{pk_A}(n_A,n_B)} \> \text{forward message} \> \sendmessageleft*{E_{pk_A}(n_A,n_B)}\\
            \text{Check }n_A \\
            \> \sendmessageright*{E_{pk_E}(n_B)} \> \text{switch keys}\>\sendmessageright*{E_{pk_B}(n_B)}\\
            \> \> \> \>\text{Check } n_B\\
            \\
            k = \textrm{KDF}(n_A,n_B) \> \> k = \textrm{KDF}(n_A,n_B) \> \> k = \textrm{KDF}(n_A,n_B) \\
        }
    }     
\end{figure}
The MITM attack to get the nonce no longer works, as $n_B$ is never encrypted under Eve's public key.
However, an adversary no longer needs to interact with Alice to impersonate her towards Bob. They are not able to get the resulting key, but break authentication as Alice did not participate.

\Exercise{9.3}
The protocol described in the exercise:
\begin{center}
    \fbox{
        \procedure{}{
            \textbf{S} \> \> \textbf{R}\\ [0.1\baselineskip][\hline]
            \\ [-0.5\baselineskip]
            \text{Choose }  n_S\\
            \> \sendmessageright*{n_S} \\%ID_A, n_A\\
            \> \>  \text{Choose } n_R\\
            \> \sendmessageleft*{E_{K_{SR}}(n_S), E_{K_{SR}}(n_R)} \>\\
            \text{Check }n_S \\
            \> \sendmessageright*{n_R} \\
            \> \> \text{Check } n_R
        }
}     
\end{center}

\begin{center}
    \fbox{
        \procedure{}{
            \textbf{A(R)} \>\> \textbf{E} \> \> \textbf{A(S)}\\ [0.1\baselineskip][\hline]
            \> \sendmessageright*{username_A} \> \> \sendmessageright*{username_B}\pclb
            \pcintertext[dotted]{protocol}\\
            \\ [-2\baselineskip]
            \> \> \> \> \text{Choose }  n_S\\
            \> \sendmessageleft*{n_S} \> \> \sendmessageleft*{n_S} \\%ID_A, n_A\\
             \text{Choose } n_R\\
            \> \sendmessageright*{E_{K_{SR}}(n_S), E_{K_{SR}}(n_R)} \> \> \sendmessageright*{E_{K_{SR}}(n_S), E_{K_{SR}}(n_R)}\\
            \> \> \> \> \text{Check }n_S \\
            \> \sendmessageleft*{n_R} \> \> \sendmessageleft*{n_R}  \\
             \text{Check } n_R
        }
}     
\end{center}
% \begin{figure}[H]
%     \begin{tabular}{cc}
%         \fbox{
%         \procedure{A starting session with B}{
%             \textbf{A(R)} \> \> \textbf{B(S)}\\  [-0.5\baselineskip]
%             \> \sendmessageright*{username_A}\pclb
%             \pcintertext[dotted]{protocol}\\
%             \> \> \text{Choose }  n_S\\
%             \> \sendmessageleft*{n_S} \\%ID_A, n_A\\
%             \text{Choose } n_R\\
%             \> \sendmessageright*{E_{K_{SR}}(n_S), E_{K_{SR}}(n_R)} \>\\
%             \> \> \text{Check }n_S \\
%             \> \sendmessageleft*{n_R} \\
%             \text{Check } n_R
%             }
%             }  
%             &
%             \fbox{
%                 \procedure{B starting session with A}{
%                     \textbf{B(R)} \> \> \textbf{A(S)}\\  [-0.5\baselineskip]
%                     \> \sendmessageright*{username_B}\pclb
%                     \pcintertext[dotted]{protocol}\\
%                     \> \> \text{Choose }  n_S\\
%                     \> \sendmessageleft*{n_S} \\%ID_A, n_A\\
%                     \text{Choose } n_R\\
%                     \> \sendmessageright*{E_{K_{SR}}(n_S), E_{K_{SR}}(n_R)} \>\\
%                     \> \> \text{Check }n_S \\
%                     \> \sendmessageleft*{n_R} \\
%                     \text{Check } n_R
%                     }
%                     }     
%                 \end{tabular}\\
% \end{figure}
\begin{itemize}
    \item 
    A is willing to play both roles in the protocol, as the key used is symmetric we can get A to decrypt messages which only B should be decrypting. 
    
    Replay the nonce from A in the other session, getting A to encrypt it along with a new nonce. Use these encyryptions to send back to A convincing them B is live.

    \item 
    Either add state and track which nonces you have chosen, or add something to the encryptions tying them to that specific session/who they are sent from.
\end{itemize}

\Exercise{9.4}
The idea here is to notice that for instance $g \in \{-1,0,1\}$ would be very poor choice since regardless of how good $x$ and $y$ are is, $g^{xy}$ in these cases can only assume very few values. If there are students which are good mathematically you can talk about groups and subgroups (if you know about that :) ). With p I am just fishing after them thinking about how this works. Of course if p is very small then $g^{xy}$ can only assume few values

\Exercise{9.6}
\begin{itemize}
    \item[(1)]
    Sending a hash of the password would allow a replay attack, this may still be the case if it is encrypted.
    A random challenge prevents this for a passive attacker, as the signature will not be useful later, even in the clear.
    \item[(2)]
    Access to the password hash would allow an offline bruteforcing.
    Access to the public key would not help the read only attacker. 
    If the attacker could change the hash or public key they could gain access to the system.
    \item[(3)] 
    The password may be in the users head, in which case access to the disk would not be a problem, an active attack may still allow stealing the password at time of login.
    If the key were stored in the clear on the disk this would it to be stolen.
\end{itemize}

\Exercise{9.7}
Making the function slow to evaluate makes a dictionary attack more expensive. 
Adding the salt prevents precomputation for passwords forcing an attacker to target a single user at a time.

\Exercise{9.8}

\begin{itemize}
    \item[(1)]
    A keylogger on the pc would allow a hacker to steal the password.
    It is probably harder to log keys on the dedicated hardware.

    \item[(2)]
    The user cannot see what they are signing, corrupt software may make them sign the wrong document.

    \item[(3)]
    This would prevent the software from signing an arbitrary number of documents for each time the user authenticates. 
    This may help the user discover if the wrong thing is singed, but they may not notice this. 

    \item[(4)]  
    This shifts the system to a single point of trust/failure. 
    There is still centralised trust of those producing the hardware units, so it is debatable how big a change this is.
    For some users it may be a better solution to have centralised authentication, especially less technically proficient users.

\end{itemize}

\Exercise{9.9}
For any user can guess $L R$ passwords.
For uniform passwords this gives success probability. 
$$ LR / S$$
We can find the probability of breaking at least one password by taking one minus the probability of breaking no passwords. 
$$ 1 - \left(1- \frac{LR}{S}\right)^N$$
\newpage

\begin{comment}
\Exercise{9.5}
For $d$ digits
$$d^\ell = 2^{128}$$
$$\ell =  128 / \log (d)$$

\begin{table}[h]
    \centering
    \begin{tabular}{c|c}
        $d$ & $\ell$\\
        \hline
        10 & 39\\
        36 & 25\\
        128 & 19\\
    \end{tabular}
\end{table}
It is worthwhile discussing the diminishing returns of a larger alphabet vs length.
Is it easier to have a longer password of non-special characters?

\Exercise{9.6}
\begin{itemize}
    \item[(3)] $E_{MK}(K \oplus CV),~ CV$ \\
    This is obviously broken for any encryption scheme.
    \item[(1)] If the ciphertext is malleable, think OTP, CTR, ... a bit of the CV and encryption can be flipped.
    \item[(2)] For an idealised encryption scheme flipping a bit of the key should make the decryption entirely random and unrelated. 
\end{itemize}

\Exercise{9.7}
Read only access is not a problem, the insider cannot see which cards have which pins.
For write only access the insider could replace the pin of a card with their own pin.
This could still work for $pin \oplus cardno$ as the insider may be able to flip the corresponding bits of their pin to cancel out the wrong card number.

\Exercise{9.8}
This places trust on the postal service. It may be possible to intercept this letter, e.g. steal the letter from the post box.
Trust needs to be established at some point, this requires some real world interaction, mail may be a reasonable choice.

The students will have surprisingly many root certificates, also from organisations that they do not necessarily trust.

\Exercise{9.10}
\begin{itemize}
    \item[(1)]
    $$ (9/10)^{40} \approx 0.015$$
    Can exclude $10^4 - 9^4 = 3439$.
    \item[(2)]
    Pincodes with many of the same numbers are more likely to be excluded.
    A single one with probability:
    $$ (9/10)^{39} (1/10)  \approx 0.0016$$
    For each position of the $1$ there are $9^3$ choices of the remaining digits. $4 \cdot 9^3 = 2916$
    Each of these cases could be considered for every possible digit.
    \item[(3)]
    Humans are very bad at choosing random numbers, there may be significant bias.
    \item[(4)]
    The quality of this solution depends on the quality of the alternative, it is better than writing the whole pin down, but worse than remembering it.
    Reducing the search space somewhat may not be an issue if the attacker only has three pin attempts and there are several thousand possibilities left.
\end{itemize}
\end{comment}